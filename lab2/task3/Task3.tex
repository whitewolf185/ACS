\begin{center}
\large{\textbf{Приведение уравнения поверхности второго рода к каноническому виду}}
\end{center}

\begin{sagesilent}
	func(x,y,z) = -x^2 + 2*x*y - 9*y^2 + 6*x*z + 18*y*z - 11*z^2 + 1
	
	A = matrix([
    [-1, 1, 3],
    [1, -9, 9],
    [3, 9, -11]
])

a0 = 1
    
\end{sagesilent}

\textit{Вариант 4}

\subsubsection{Функция}
$$f = -x^2 + 2*x*y - 9*y^2 + 6*x*z + 18*y*z - 11*z^2 + 1$$

\subsubsection{Матрица квадратичной формы}
\begin{equation*}
A = \sage{A}, \quad a_0 = \sage{a0}
\end{equation*}

\subsubsection{Характеристический многочлен}
\begin{sagesilent}
# Характеристический многочлен
var('l', latex_name=r"\lambda")
char_poly = (A - matrix.identity(3) * l).det().simplify_full()
\end{sagesilent}
$$|A - \lambda * E| = \sage{char_poly}$$

\subsubsection{СЗ}
\begin{sagesilent}
# Собственные значения
eigen_values = [i.rhs().real() for i in solve(char_poly == 0, l)]
eigen_values.sort()
\end{sagesilent}
$$\lambda_1 = \sage{eigen_values[0].n(digits = 5)},$$
$$\lambda_2 = \sage{eigen_values[1].n(digits = 5)},$$
$$\lambda_3 = \sage{eigen_values[2].n(digits = 5)}$$

    

\subsubsection{СВ}
\begin{sagesilent}
# Собственные вектора
eigen_vectors = []
var("x1 x2 x3")
for i in range(len(eigen_values)):
    char_poly = A - eigen_values[i] * identity_matrix(3)
    system_of_equations = []
    for j in range(3):
        system_of_equations.append(char_poly[j][0] * x1 + char_poly[j][1] * x2 + char_poly[j][2] * x3 == 0)
    # Решаем систему уравнений методом Гаусса
    system_of_equations[1] = system_of_equations[1] - system_of_equations[0] * char_poly[1][0] / char_poly[0][0]
    system_of_equations[2] = system_of_equations[2] - system_of_equations[0] * char_poly[2][0] / char_poly[0][0]
    system_of_equations[2] = system_of_equations[2] - system_of_equations[1] * (system_of_equations[2].lhs() / system_of_equations[1].lhs())
    # Проверяем, что ранг матрицы меньше трёх
    if (system_of_equations[2].lhs() == 0 and system_of_equations[2].rhs() == 0):
        system_of_equations[2] = (x1 == 1)
    else:
        print("Ранг матрицы равен трём")
    eigenvec = vector([j.rhs().n(digits = 6) for j in solve(system_of_equations, x1, x2, x3)[0]])
    ans = char_poly * eigenvec
    eigen_vectors.append(eigenvec.n(digits = 6))
\end{sagesilent}
$$s_1 = \sage{eigen_vectors[0]}$$
$$s_2 = \sage{eigen_vectors[1]}$$
$$s_3 = \sage{eigen_vectors[2]}$$


\subsubsection{Матрица перехода из нормированых СВ}
\begin{sagesilent}
# Матрица перехода из нормированых собственных векторов
S = list()
for ev in eigen_vectors:
    S.append(ev / sqrt(ev * ev))
S = matrix(S).T
\end{sagesilent}
$$S = \sage{S.n(digits = 5)}$$

\subsubsection{Диагональная матрица}
\begin{sagesilent}
# Диагональная матрица
A1 = S.T * A * S
\end{sagesilent}
$$\Lambda = S^T * A * S = \sage{A1.n(digits = 5)}$$

\subsubsection{Приведенное ур-ние}
\begin{sagesilent}
# Приведенное уравнение
v = a0
variables = [x, y, z]
for i in range(len(variables)):
    v += A1[i][i] * variables[i] ^ 2
\end{sagesilent}
$$\sage{v==0}$$

\subsubsection{График канонического уравнения}
\begin{center}
  \sageplot[trim=100 100 100 100, clip, width=3in][png]{implicit_plot3d(v, (x, -10, 10), (y, -20, 20), (z, -20, 20), plot_points=100)}
\end{center}

\subsubsection{Исходный график}
\begin{center}
  \sageplot[trim=100 100 100 100, clip, width=3in][png]{implicit_plot3d(func, (x, -50, 50), (y, -50, 50), (z, -50, 50), plot_points=200)}
\end{center}